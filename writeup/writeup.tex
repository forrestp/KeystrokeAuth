%
%   latex writeup for KeystrokeAuth
%   6.858 Final Project
%   December 2013
%   forrestp, ameeshg, kseibert, cbieden
%
%

\documentclass{article}

\usepackage{geometry}
\geometry{letterpaper}

\usepackage{doc}

\usepackage{graphicx}

\usepackage{epstopdf}


\title{KeystrokeAuth}
\author{
  Forrest Pieper\\
  Carlo Biedenharn\\
  Kenneth Seibert\\
  Ameesh Goyal
}
\date{December 10th, 2013}

\begin{document}

\maketitle

\abstract{
}

\section{Introduction}
\label{introduction}
KeystrokeAuth is an example website implementation that uses keystroke timing to provide stronger user authentication.
Measuring keystroke timing is a method for passive biometric authentication. 
Traditional biometric authentication such as fingerprint or retinal scanners require external hardware and is not suited for web service authentication where users may login from a variety of machines. 
Keystroke timing can be gathered using javascript embedded in the login and registration pages.
Thus this method requires no additional hardware.
The user must enter her password several times during registration instead of just once or twice, but other than that this method is completely unobstrusive.
Authenticating passwords with keystroke timing makes it more difficult for an attacker who possesses a user's plaintext password to compromise the account.
Additionally, it discourages account sharing which may be useful for highly secure systems and premium acconts.
In this paper we describe past work on the topic, introduce our example implementation called KeystrokeAuth, analyze the added security of our system, and examine a small set of test data.

\section{Background and Related Work}
TODO: forrest / carlo / kenny / ameesh

describe various features
\begin{enumerate}[-]
  \item flight:
  \item dwell:
  \item down-down:
  \item up-up:
\end{enumerate}

describe various detectors

\section{KeystrokeAuth Implementation}
KeystrokeAuth uses javascript to capture the timestamps on each keydown and keyup event while typing the password.
During registration, the user enters her password 10 times.
The data is sent to the server and KeystrokeAuth computes a model specific to that user and password.
When logging in, the user enters the password once and KeystrokeAuth compares the new timing data to the registered model.
If the timing data differs by too much, the user will not be logged in.

\subsection{Gathering Timing Data}
TODO: forrest

Gather code, uptime, downtime --> compute various features

\subsection{Generating User Timing Model}
TODO: ameesh / kenny  ***make this part feature-agnostic

\subsection{Login Timing Authentication}
TODO: ameesh / kenny  ***make this part feature-agnostic 

\section{Security Analysis}
TODO: forrest

proof that security is not worse

strategy: make it no less convinient/difficult for users, and at least slightly more secure

\section{Data Collection and Analysis}
TODO: carlo

\subsection{Data Overview}
% how you gathered data
% graphs demonstrating biometric uniqueness

\subsection{Feature Comparison} %dwell vs flight vs down-down etc.
%table of which features they pass both detectors 


\subsection{Error Rates}    %false positive (someone else logs in) and false negative (I can't log in)

Goal: Find ideal thresholds and weights for each feature that give 1\% false negative rate
ideal would be  .001\% false positive, but anything less than 100\% is an improvement over state of the art
Compromise: increase false positive rate to accomodate 1\% false negative

\section{Conclusion}
TODO: forrest / carlo / kenny / ameesh


\begin{thebibliography}{99}
  \bibitem{Killourhy09}
    %comparison of various keystroke timing schemes
   Killourhy, Kevin S., and Roy A. Maxion. 
   ``Comparing anomaly-detection algorithms for keystroke dynamics.''
   \textit{Dependable Systems \& Networks, 2009. DSN'09. IEEE/IFIP International Conference on.}
   IEEE, 2009. 
 
 \bibitem{Cho00}
    %Nearest-neighbor mahalanobis
   Cho, Sungzoon, et al.
   ``Web-based keystroke dynamics identity verification using neural network.'' 
   \textit{Journal of organizational computing and electronic commerce}
   10.4 (2000): 295-307.
  
\end{thebibliography}

\end {document}
