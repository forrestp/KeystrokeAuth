%
%   latex writeup for KeystrokeAuth
%   6.858 Final Project
%   December 2013
%   forrestp, ameeshg, kseibert, cbieden
%
%

\documentclass{article}

\usepackage{geometry}
\geometry{letterpaper}

\usepackage{doc}

\usepackage{graphicx}

\usepackage{epstopdf}


\title{KeystrokeAuth}
\author{
  Forrest Pieper\\
  Carlo Biedenharn\\
  Kenneth Seibert\\
  Ameesh Goyal
}
\date{December 10th, 2013}

\begin{document}

\maketitle

\abstract{
KeystrokeAuth is an example implementation of using keystroke timing as a method of authenticating users.
It is a primary example of passive biometric authentication. 
Traditional biometric authentication such as fingerprint or retinal scanners require external hardware and are not suited for web service authentication where users may login from a variety of machines. 
Keystroke timing can be gathered during the standard login password, requiring no additional hardware and 
}




\begin{thebibliography}{99}
  \bibitem{Killourhy09}
    %comparison of various keystroke timing schemes
   Killourhy, Kevin S., and Roy A. Maxion. 
   ``Comparing anomaly-detection algorithms for keystroke dynamics.''
   \textit{Dependable Systems \& Networks, 2009. DSN'09. IEEE/IFIP International Conference on.}
   IEEE, 2009. 
 
 \bibitem{Cho00}
    %Nearest-neighbor mahalanobix
   ``Web-based keystroke dynamics identity verification using neural network.'' 
   \textit{Journal of organizational computing and electronic commerce}
   10.4 (2000): 295-307.
  
 \bibitem{test}
    Forrest Pieper,
    Some paper.
    Publisher
    Edition
    date

\end{thebibliography}

\end {document}
